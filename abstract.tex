\begin{center}
  \large{\textbf{Abstract}}
\end{center}

\selectlanguage{english}

The union of computational, measurement and communication systems applied to electrical power systems will form the Smart Grids. One of its main components are the Smart Meters, which are measurement systems that periodically measure the consumed power in electrical loads, and that send the measurements to the electricity companies. The research, and mainly the installation of smart meters is more developed in the north hemisphere countries, although it already started in Brazil. When this measures become ubiquitous in the Brazilian electrical power system, a considerable amount of data will be produced, and the extraction of useful information from it will be necessary. One of the possible data mining tasks that could be applied to this data, and which is the object of this dissertation, is the clustering of load curves, from which is expected to get information and knowlodge that can enhance the distribution service, the load forecast , the planning of expansion, and others. As the data of the smart meters installed in the Brazilian electrical system is not publicly available, the load curves of Australian residences were clustered. Since that, in this country, the climatic variations that could influence in the consuming profile are very similar to the climatical brazilian variations. The problem of clustering load profiles is a time series clustering problem, since that for each measurement there is the temporal component associated. Thus, in this master dissertation was presented, initially, a study of clustering of time series, and this study is focused in the preprocessing, dissimilarity measure and algorithm applied, as well as the method of evaluating the results. Initially, after an empirical study in labeled datasets, the general clustering of time series strategy was proposed, and it is composed by: Z normalization in the preprocessing step, CID-DTW as the dissimilarity the hierarchical clustering algorithm using the average linkage and the silhouette And Calinski-Harabasz index for the evaluation of the obtained partitions. In tha case study, with the australian dataset, such strategy generated highly unbalanced partitions, which motivated the proposition of a multiobjective approach, in which partitions with a good compromise between the validation index values, wich is a measure of the desired low variation intragroup and high variance intergroup, and balanced partitions, in which the variance of the number of instances in each cluster is low. The multi-objective approach indicates the tha max normalization, the k-means algorithm and the euclidean distance is the most appropiate configuration for the clustering of load profiles generated by the smart meters.

\vspace{.5cm}
\textbf{Keywords}:
% FIXME Remover a linha abaixo.
Clustering, time series, load curves, smart meters.
% TODO Inserir as palavras-chave em inglês aqui.
\selectlanguage{brazilian}
