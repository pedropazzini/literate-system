\begin{center}
  \large{\textbf{Resumo}}
\end{center}

% TODO Inserir o resumo em português aqui.
A união de sistemas computacionais, de medição e de comunicação aplicada aos sistemas elétricos de potência dará forma às chamadas Redes Inteligentes. Um de seus componentes principais são os chamados medidores inteligentes, que são instrumentos que medem periodicamente a potência consumida em cargas elétricas, e enviam estes dados de medição para as distribuidoras de energia. A pesquisa, e principalmente a instalação de medidores inteligentes é mais intensa nos países do hemisfério norte, no entanto ela já se iniciou também no Brasil. A partir do momento em que esses medidores se tornarem onipresentes no sistema elétrico de potência brasileiro, uma massa de dados considerável será gerada, e se fará necessária a extração de informação útil dela. Uma das possíveis tarefas de mineração de dados a ser realizada nesta massa de dados, e que será objeto desta dissertação de mestrado, será o agrupamento das cargas, a partir do qual se espera obter informações e conhecimento que possa aprimorar o serviço de distribuição, a previsão de demanda, o planejamento de expansão da rede, dentre outras possibilidades. Uma vez que ainda não estão disponíveis os dados dos medidores inteligentes instalados no Brasil, foi realizado o agrupamento de curvas de carga medidas em residências australianas, já que neste país, as variações climáticas que podem influenciar no consumo de energia elétrica são bem similares às variações climáticas brasileiras. O problema de agrupamento de curvas de carga, é um problema de agrupamento de séries temporais, já que a cada medição realizada pelos medidores inteligentes existe uma componente temporal associada. Assim, nesta dissertação de mestrado foi apresentado, inicialmente, um estudo das estratégias de agrupamento de séries temporais no que diz respeito ao pré-processamento empregado, métrica de dissimilaridade e algoritmo de agrupamento utilizados, bem como o método de avaliação dos resultados obtidos. Inicialmente, em um estudo empírico em base de dados rotuladas, foi definida uma estratégia genérica para o agrupamento de séries temporais, composta por: normalização Z na etapa de pré-processamento, escolha da métrica de dissimilaridade CID-DTW e o algoritmo hierárquico com \emph{linkagem average} para a realização do agrupamento em si, e os índices de validação silhouette e Calinski-Harabasz para a avaliação das partições obtidas. No estudo de caso, com os dados australianos, tal estratégia gerou grupos altamente desbalanceados, o que motivou a proposição de uma abordagem multi-objetiva, na qual se procura obter partições com um bom índice de validação, que por sua vez indica a tão desejada baixa variância intragrupo e alta variância intergrupo, e partições balanceadas, nas quais o número de instâncias em cada grupo possui baixa variância. A abordagem multi-objetiva indica que a normalização \emph{max}, aliada ao algoritmo \emph{k-means}, fazendo uso da distância euclidiana como métrica de dissimilaridade é a configuração mais adequada para o agrupamento de curvas de carga geradas por medidores inteligentes.

\vspace{.2cm}
\textbf{Palavras-chave}:

% TODO Inserir as palavras-chave aqui.
Agrupamento, séries temporais, curvas de carga, medidores inteligentes.